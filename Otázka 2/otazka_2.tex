\documentclass[11pt]{article}
\usepackage[utf8]{inputenc}
\usepackage{a4wide}
\usepackage{amsmath, amsfonts, amssymb}
\usepackage{graphicx}

\usepackage{multicol}
\setlength{\parindent}{0pt}
\def\dline#1{\underline{\underline{#1}}}

\begin{document}

\title{Otázka 2 - Princip práce počítače, schéma počítače, umělá inteligence.}
\author{Ondřej Nevěřil}
\date{25. 9. 2025}
\maketitle

\section*{Schéma počítače}

PC -- zařízení pro zpracování informací

vstup $\to$ výstup

PC -- hardware (technické vybavení) + software (programové vybavení)

FW (firmware) -- SW zabudovaný v HW, není běžně měnitelný, dá se aktualizovat\\

Schéma PC -- vychází z Johna von Neumanna:\\

Vstupní informace $\rightarrow$ Vstupní zařízení $\rightarrow$ RAM (+ ROM), disky, CPU $\rightarrow$ Výstupní zařízení $\rightarrow$ výstupní informace\\

\begin{description}
\item[ROM] -- read only memory -- vestavěná paměť -- jak startovat počítač, dlouhodobá paměť
\item[RAM] -- random acces memory -- krátkodobá paměť -- ukládá data spuštěných aplikací, když se vypne PC, tak se vymaže
\item[CPU] -- Řídí činnost PC, má více jader a vláken, každé řádově $GHz$ operací. Obsahuje:
\begin{description}
\item[Řadič] -- řídí činnost
\item[ALJ] -- aritmeticko logická jednotka - výpočty
\item[Registry] -- krátkodobá paměť, aby CPU nemusel do RAM- počet $16$ a více
\end{description}
\item[Registry v OS] -- informace potřebné k správnému chodu OS
\item[Cache paměť] -- dočasné úložiště -- vyrovnává rychlost mezi CPU a RAM
\end{description}

\section*{Bity, byty}

$1$ bit $= 1b = $ nejmenší jednotka informace -- ano nebo ne, fyzikálně lze snadno realizovat

Použití jednotky např. rychlost sítí

$1$ byte $= 1B = 8b$, čísla v intervalu $<0, 255>$

Použití -- ASCII tabulka -- pro každou možnost jeden znak

\section*{Tiskárny}

\begin{enumerate}
\item Jehličkové -- drahé, hlučné, malé DPI, levný tisk, pouze černě
\item Laserové a LED -- černý i barevný tisk, kvalitní, čistá výměna tonerů,  vysoké provozní náklady.
\item Inkoustové tiskárny - barevný tisk, kvalitní, rychlé, levné
\item termální tiskárny $\cdots$
\item $3$D tiskárny
\end{enumerate}

Rozdělení:
\begin{itemize}
\item Tiskárny
\item Multifunkce - scanner - scanování, kopírování
\item (Fax)
\end{itemize}

Připojení
\begin{itemize}
\item USB
\item LAN -- výhodné pro více PC v síti
\item Wifi
\item Bluetooth
\end{itemize}

\end{document}