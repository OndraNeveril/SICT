\documentclass[11pt]{article}
\usepackage[utf8]{inputenc}
\usepackage{a4wide}
\usepackage{amsmath, amsfonts, amssymb}
\usepackage{graphicx}

\usepackage{multicol}
\setlength{\parindent}{0pt}
\def\dline#1{\underline{\underline{#1}}}

\begin{document}

\title{Otázka 5 - Operační systémy, konfigurace pc, diskové manažery, údržba pc}
\author{Ondřej Nevěřil}
\date{23. 10. 2025}
\maketitle

\section*{Operační systém}

$=$ program (balík programů) nutný pro práci PC

Zajišťuje:

\begin{itemize}
\item Komunikaci mezi HW, aplikacemi a uživatelem
\item Správu systémových zdrojů -- CPU, RAM, disků
\item Správu a spouštění dalších programů
\end{itemize} 

Existují vrstvy OS:

\begin{enumerate}
\item Příkazový interpret -- GUI (Graphical user interface)/Příkazový řádek (CMD)
\item Jádro OS -- BDOS (Basic disk operating system) (1981 MSDOS)
\item BIOS (Basic input/output system) -- Firmware -- zajišťuje na nejnižší úrovni komunikaci s HW, obsahuje ovladače (drivery) zařízení
\end{enumerate}
 
Po startu PC se provede test HW (Post) a zavedení OS z disku do RAM -- Bootování, lze nastavit v BIOSu, nebo v Boot menu

Klasifikace OS:
\begin{enumerate}
\item Řádkové (MSDOS)/grafické (Mint, Windows...)
\item Jednouživatelské (MSDOS)/Síťové (lze připojit do sítě) -- pracovní stanice/server
\item jednoúlohové (MSDOS)/Multitasking(Mint, Windows...)
\end{enumerate}
\begin{center}
\includegraphics[scale=0.3]{1.png}
\end{center}

\section*{Souborový systém}
Způsob uložení a organizace dat na disku

\begin{enumerate}
\item Fyzický disk -- dnes SSD
\item Logické disky -- oddíly, partition
\item Složky (adresáře) -- obsahuje soubory, které k sobě nějak patří
\item Soubory -- pojmenovaná množina dat, jméno.přípona (podle přípony typ souboru a způsob, jak otevřít), datové (např..docx, .jpg, ...)/programy (např. .exe) -- postup pro PC, co má dělat\\ Soubory mohou mít atributy -- ARHS (archivní, jen pro čtení, skryté, sestémové) -- jméno, koncovka, datum vzniku, změny
\end{enumerate}

\end{document}