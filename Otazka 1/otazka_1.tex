\documentclass[11pt]{article}
\usepackage[utf8]{inputenc}
\usepackage{a4wide}
\usepackage{amsmath, amsfonts, amssymb}
\usepackage{graphicx}

\usepackage{multicol}
\setlength{\parindent}{0pt}
\def\dline#1{\underline{\underline{#1}}}

\begin{document}

\title{Otázka 1 - Informace a informatika. Ergonomie, ochrana zdraví při práci s počítačem}
\author{Ondřej Nevěřil}
\date{5. 9. 2025}
\maketitle

\section*{Informace, informatika}

\begin{description}

\item[Informace] - něco, co snižuje míru naší neurčitosti (entropie = nahodilost, neurčitost)
\item[Informatika] - Vědní obor, jehož předmětem zájmu je informace:

\begin{itemize}
\item Získání informací
\item Uchování informací
\item Zpracování informací
\item Poskytování informací
\item Bezpečnosz informací
\end{itemize}

\item[Počítač] - zařízení pro automatizované (pracuje podle programu) zpracování informací
\end{description}

\subsection*{Historie zpracování informace}
\begin{enumerate}
\item Řeč
\item Písmo
\item Knihtisk - Gutenberg cca $1450$ $\Rightarrow$ Zlevní se přenos informací
\item Počítač - $40.$ léta $20.$ století
\item PC = osobní počítač - $1981$
\item Internet
\item Mobilní internet
\item Mobilní zařízení
\item AI = umělá inteligence
\end{enumerate}

\subsection*{Aplikace informatiky}
\begin{itemize}
\item Počítače
\item Telekomunikace
\item Umělá inteligence
\item Expertní systémy
\item Počíačové simulace
\item Geoinformatika
\item $\cdots$
\end{itemize}

\subsubsection*{Zdroje informace}
Dělení:
\begin{enumerate}
\item Podle smyslů
\begin{itemize}
\item Vizuální
\item Audio
\item $\cdots$
\end{itemize}

\item Podle masovosti
\begin{itemize}
\item Dialog
\item Skupina
\item Masová média (televize, internet)
\end{itemize}

\item Podle média
\begin{itemize}
\item Papírová média
\item Televize
\item Internet
\end{itemize}

\item Podle periodicity
\begin{itemize}
\item Denní
\item Týdenní
\item Měsíční
\item $\cdots$
\end{itemize}

\end{enumerate}

\subsection*{Výběr informací}
\begin{enumerate}
\item Správnost
\item Objektivnost
\item Odbornost
\item Aktuálnost
\item Ucelenost, rozmanitost
\end{enumerate}

\subsection*{Vědecké informace}
\begin{enumerate}
\item Impaktový faktor
\item Recenze
\item Stárnutí informace
\end{enumerate}

\subsection*{Dostupnost informací, ochrana informací}
\begin{itemize}
\item \textbf{Zákon č.} $\mathbf{106/1999}$ \textbf{Sb.} - o svobodném přístupu k informacím
\item \textbf{Zákon č.} $\mathbf{101/2000}$ \textbf{Sb.} - na ochranu osobních údajů
\item \textbf{GDPR}  - nařízení EU o ochraně osobních údajů
\item \textbf{Zákon č.} $\mathbf{121/2000}$ \textbf{Sb.} - autorský zákon
\end{itemize}

\section*{Ergonomie práce na PC}
Ergonomie se zabývá pracovními návyky a pracovním prostředím s cílem, aby práce byla bezpečná, efektivní a méně namáhavá.\\

Rizika, problémy, řešení:
\begin{itemize}
\item Dlouhotrvající práce na PC, "strnulá" práce, hodiny pozorujeme obrazovku
\item Pohybové problémy:
\begin{enumerate}
\item Bolesti zad v bederní oblasti
\item Krční páteř
\item Zánět karpálního tunelu - řešení - ergonomická myš, podložka, powerball
\end{enumerate}
\item Problémy s očima - "pocit suchého oka" (pálení očí) - málo mrkání
\item Psychosociální problémy (závislost)
\end{itemize}

\begin{center}
\includegraphics[scale=0.3]{a}
\end{center}

Doporučení:
\begin{enumerate}
\item Doba strávená u počítače
\item Pohyb
\item Ergonomie pracoviště
\end{enumerate}

\end{document}